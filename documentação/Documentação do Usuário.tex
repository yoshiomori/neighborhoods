\documentclass[a4paper, 12pt]{article}
\usepackage[brazil]{babel}
\usepackage[utf8]{inputenc}
\usepackage{listings}

\begin{document}
\section{Uso}
Modo de uso:
\begin{lstlisting}
  neighborhood nome_arquivo
\end{lstlisting}

Onde o arquivo na entrada do programa deve estar no seguinte formato:
alfabeto\\
linha em branca\\
constante\\
linha em branca\\
matriz de dados\\
linha em branca\\
matriz de vizinhança\\
\\
Além disso é possivel fazer comentários no arquivo de entrada com \#, tudo que estiver após \# será ignorado.

O programa neighborhood irá criar um arquivo out.txt com o resultado da vizinhança para cada vertice e a probabilidade condicional para cada palavra.

\section{Limitação}
O número de vizinhos para cada vértice não pode passar de 32.

\end{document}
