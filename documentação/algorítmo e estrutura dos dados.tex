\documentclass[a4paper, 12pt]{article}
\usepackage[brazil]{babel}
\usepackage[utf8]{inputenc}
\usepackage{algorithm2e}
\usepackage{amsfonts}
\usepackage{listings}

\begin{document}

\section{Algoritmos}

O programa Neighborhood está basicamente dividido em três etapas, leitura do arquivo de entrada, processamento dos dados e a impressão.

\begin{algorithm}
  \caption{Neighborhood}
  \SetAlgoLined
  \KwData{Arquivo no formato pré estabelecido}
  \KwResult{Arquivo out.txt contendo}
  Leitura do arquivo de dados\;
  Processamento\;
  Impressão\;
\end{algorithm}

Mas a parte importante do programa é o Processamento que irá fazer o calculo da máxima verossimilhança para cada vértice a fim de encontrar sua vizinhaça.

\begin{algorithm}
  \caption{Máxima verossimilhança}
  \label{alg:a2}
  \SetAlgoLined
  \SetKwInOut{Input}{input}\SetKwInOut{Output}{output}
  \Input{Vértices(V), vizinhos de cada vértice(N(v)) e constante(c)}
  \Output{Vizinhos de cada vertice que maximizam a função verossimilhança}
  \For{$v\in V$}{
    Max $\longleftarrow $ verossimilhança(v, $\emptyset$)\;
    \For{$N\in 2^{N(v)}\setminus \emptyset$}{
      Cur $\longleftarrow $ verossimilhança(v, $N$)\;
      \If{$Cur > Max$}{
        Max $\longleftarrow $ Cur\;
      }
    }
    N(v) $\longleftarrow $ Max;
  }
\end{algorithm}

Esses são os algorítmos principais do programa neighborhood.

\section{Estrutura de dados}
Para o conjunto dos Vértices V:
\lstset{language=C}
\begin{lstlisting}
  typedef struct VERTICE *Vertice;
  Vertice vertice;
\end{lstlisting}

Onde cada vértice é:

\begin{lstlisting}
  struct VERTICE{
    int line;
    char **info;
    unsigned long int size_info;
    Neighborhood neighborhood;
  };
\end{lstlisting}

Pode-se notar que para cada vértice há um campo chamado Neighborhood que obviamente é usada para para guardar o conjunto da vizinhaça desse vértice. Sua estrutura é assim:

\begin{lstlisting}
  typedef struct NEIGHBORHOOD{
    Vertice *vertice;
    int size;
    word_table table;
  }Neighborhood;
\end{lstlisting}

O Max e Cur do Algoritmo~\ref{alg:a2} estão representados nessa estrutura:

\begin{lstlisting}
  word_table cur, prev;

  typedef struct WORD_TABLE{
    Vertice *set;
    node word_vert_neig;
    int length_vert_neig;
    int free_vert_neig;
    node word_neig;
    int length_neig;
    int free_neig;
    
    double f;
  }*word_table;
\end{lstlisting}

onde Max é o prev na implementação.

E no calculo da verossimilhança foi usado a estrutura linear probing para a tabela de palavras.

\end{document}
